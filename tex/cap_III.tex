\chapter{Related Works}
\label{cap:RW}


Social network analysis is a field in the study of sociology to enable methods, techniques and models to be developed to help researchers understand individuals' behavior and relationships. Graph theory has been widely used to aid in the identification and understanding of social network structures. Research conducted over the years has shown that social networks have peculiar structures that are not found in random graphs.

In 1969, Travers and Milgramm \cite{Travers:1967} conducted an experiment involving social relationships to evaluate the concept of the "small world" they had formulated years earlier. The results showed that the analyzed social network exhibits a high number of shortest paths, a structure that was found not only in other social networks but also in biological and technological networks by Watts and Strogatz (1998) \cite{Watts1998}. The fascination with these findings is revealed in the fact that this structure is not found in random graphs.

The organization of individuals in cohesive groups and the impact that the relationship can cause in the process of information diffusion was explored in 1973 by Granovetter \cite{Granovetter1973}, where the he concludes that the relationship is modeled in order to show strong bonds between individuals, forming communities, and weak ties that allow the interaction and diffusion of information between communities. Communities are also not found in random graphs.

Barabasi and Albert (1999) \cite{Barabasi509} conducted a study that presented a common characteristic found in several complex systems, among them some social networks, that is the fact of vertices connectivity following a scale-free power-law distribution. The authors state that this characteristic is a consequence of network expansion associated with the concept of "preferential connection", in which new vertices tend to connect with other vertices already well connected.

With the advent of the OSN, and consequently with the increasing amount of information present in these environments, researchers from different areas of knowledge were interested in the analysis of the structure of these networks. In 1997, Garton et al. \cite {Garton1997} developed a work to identify how online social networks could be analyzed and how virtual relationships would interfere with the functioning of social systems. The authors show that the analysis and description of relationships can be done with an ego approach - when the network is too large or difficult to define - as well as through a global analysis - when the entire network is available. Some characteristics of the OSN that should be carefully analyzed, according to the authors, are: size (hence the heterogeneity of individuals), structural position of individuals, similarity of behavior of individuals and formation of groups. The authors also report that social networks need to be explored from a multilayered view, but so far there have been few studies that explore this last item.

In the work of Ahn et al. (2007) \cite{Ahn2007}, the authors compare the structure of three OSNs: Cyworld, MySpace and Orkut. The properties analyzed were the distribution of degrees, grouping properties, correlation of degrees and evolution over time. The only complete OSNs was Cyworld, and for the other networks small samples were used. The results for MySpace and Orkut have been very close, while the Cyworld network differs from the others and seems to achieve a stabilization over time. The authors also showed that the analyzed OSNs differ from real-life social networks in the analyzed aspects.

Mislove et al (2007) \cite{Mislove2007} also analyzed the structure of different OSNs: Flickr, YouTube, LiveJournal, and Orkut. The authors concluded that the analyzed OSN have the following properties: scale-free power-law degree distribution and small-world phenomenon. the authors also report that node input degrees tend to approach output degrees and that networks have core of vertices with high densely connected degrees that attach to small groups of vertices with low degrees but strongly grouped. The same properties also reported in the work of Fu et al. (2008) \cite{Fu2008}, who analyzed a blog and a Chinese social network.

So far we have presented only works that analyze OSN structures using only a specific type of interaction. Current OSNs allow their users to use various forms of interaction, which may have different meanings and intensities, characterizing them as multilayer networks. Our proposal is to analyze structures present in the various layers of Twitter's ego networks. The ego approach will be used taking into account the advantages presented in the Introduction of the dissertation. In section \ref{sec:ego_net_analysis} we present some of the main works that analyze the structural properties of ego networks, and in section \ref{sec:multilayer_net_analysis} we present some of the main works that analyze the structural properties of multilayer networks.

%%%%%%%%%%%%%%%%%%%%%%%%%%%%%%%%%%%%%%%%%%%%%%%%%%%%%%%%
%%%%%%%%%%%%%%%%%%%%%%%%%%%%%%%%%%%%%%%%%%%%%%%%%%%%%%%%


\section{Structural Analysis of Ego Network}
\label{sec:ego_net_analysis}

The comparison between ego network properties in OSN and social networks in the real world, or offline networks, has been widely investigated. Once researchers are able to compare these two environments, the analysis of ego networks can be performed more efficiently, because in the OSN it is much more practical to collect the relationship data. As most researches have shown several similarities between the two environments as in \cite{Gala2012,Arnaboldi:2012,Arnaboldi:2013,Arnaboldi:2013a,Arnaboldi:2016,Arnaboldi:2017a,ARNABOLDI2017}, the OSNs become essential tools for understanding human behavior. However, as we can observe in the following work description, the vast majority of them consider only the comparison of structures present in OSN with well-known patterns in the study of offline social networks, and that a great amount of information may be being discarded. We conjecture that the use of appropriate measures and modeling for graph analysis can shed light on patterns not yet explored in the OSN ego networks.

In Arnaboldi et al. (2013) \cite{Arnaboldi:2013}, analyzes show that Twitter presents social structures qualitatively similar to those found by the authors themselves on Facebook and by Dunbar in ego offline networks. The modeling of Twitter's ego networks was accomplished by aggregating the follow, retweet, mention, and replies interactions indistinctly. The comparison was made by identifying similarity between the pattern found in offline ego networks known as Dunbar circles, and the Twitter's ego networks that they modeled. The results suggest that the structure of a Twitter's ego network is controlled by the same cognitive properties of the human brain operating in offline ego networks.

The behavior of the users in different OSN was studied in Arnaboldi et al. (2013a) \cite{Arnaboldi:2013a}. Here the authors also perform a comparison between ego networks in OSN and offline ego networks in the sense of analyzing the dynamic processes of ego networks and personal social relations on Twitter. The authors suggest that human behavior on Twitter differs significantly from other social networks studied in the literature in different fields of research. Twitter ego networks are generally smaller and have a high percentage of weak and dynamic links, which culminates in a difficulty in determining that there is a hypothetical decline in Twitter usage, unlike in other OSNs.

Already in Arnaboldi et al. (2016) \cite{Arnaboldi:2016} and Arnaboldi et al. (2017) \cite{ARNABOLDI2017}, the authors analyze the behavior of users in online and offline ego networks to identify common patterns their relationship with the process of information diffusion. The authors state that this understanding can be useful in creating new services for the Internet of the Future, such as highly personalized ads, tailored to the needs and characteristics of each user. The results point to the similarity in the behavior of online and offline ego network users during the process of information diffusion.

In Arnaboldi et al. (2017a) \cite{Arnaboldi:2017a}, the authors state that the evolution of the bonds between ego and alters is very dynamic, suggesting that they are not entirely used for social interaction, but for public signposting and self-promotion.

In the works described so far that compare the structures found in ego networks in the OSN and ego networks in offline social networks, the authors model star ego networks, that is, they considered only the relationship between ego and alter, ignoring the alters-alters edges. Another important point is that they also disregard the meaning of each interaction, forming an aggregate network, not evaluating the importance of each interaction in the aggregate network formation process. Our proposal differs from these works precisely by modeling ego networks considering the meaning and importance of each type of interaction and considering the alters-alters edges.

The analysis of ego networks usually happens when you do not have enough data to model the entire network. However, the results obtained in analyzes of ego networks show that they may be different from the results obtained when the entire network is analyzed. This fact shows that an ego network analysis should be considered only when the intent is to actually observer user behaviors in a local approach, otherwise the results may be distorted.

Gupta et al. (2015) \cite{Gupta2015} analyzed different OSNs (social, co-authoring, communication and hybrid) and show that the structural properties of ego networks are different from the entire network. For this the authors analyzed the distribution of degrees, degree of assortativity and clustering coefficient of ego networks derived from each considered OSN and compared with the same metrics calculated across the network. These results help to better understand and correct the biases resulting from insufficient local information in the OSNs, thus allowing an effective analysis of social behaviors.

The social networks analyzed by Gupta et al. were Facebook and Orkut, and the modeling of ego networks is accomplished through the explicit friendship relationship between users on these OSN. Our local approach takes into account the results presented by Gupta et al. and in our analysis we are focused in studying the characteristics of an individual and his/her social environment as a way to detect repetitions of patterns at different points in the whole network and to open the way for works that improve the offer of personalized services.

In the work of \cite{Muhammad2015}, the authors characterized structural patterns of alters in nine publicly available spatial-temporal, collaboration and social ego networking datasets. Using graph mining techniques, the authors extracted features that can be classified according to centrality, efficiency, transitivity, and actor-based, and applied unsupervised clustering techniques to identify and categorize eight ego neighbors' clustering patterns still in connected structures, dense structures, informative ego and less dense structures. The authors also analyzed the behavior of the alters by comparing the patterns found by the same set of users in three different OSNs. Here the authors only make a structural characterization of ego networks, not an analysis as we propose.

In the work of Achiam et al. (2016) \cite{Achiam2016}, the authors analyzed the ego network structure of followers of some Twitter profiles associated with companies, ie using the followee interaction. They show that the distribution of the ego network does not follow a power-law, that is, it is not scale-free and is not random. From then on they present a model for capturing the dynamics of these ego networks specifically focused on companies and the type of relationship derived from the followee interaction. Here the authors modeled and analyzed specific ego networks through a single type of interaction. We are interested to find out if there are common patterns in ego-modeled networks from different points of Twitter. For this we analyze several types of interaction between the vertices of several ego networks.
%%%%%%%%%%%%%%%%%%%%%%%%%%%%%%%%%%%%%%%%%%%%%%%%%%%%%%%%
%%%%%%%%%%%%%%%%%%%%%%%%%%%%%%%%%%%%%%%%%%%%%%%%%%%%%%%%

Among the papers that seek to analyze structural patterns present in Twitter ego networks, there are those who dedicate themselves to detect communities \cite{McAuley:2012,Leskovec2012,Dykstra2014,Epasto2015}. Girvan and Newman (2003) \cite{Girvan2003} define a community as a subset of the set of network vertices that are densely connected to each other and have low connection density with the other network vertices not belonging to this subset. This is the widely used definition, although there are other models and other nomenclatures \cite{Lancichinetti2009b, Lancichinetti2011, Leskovec2012}.

Soundarajan and Hopcroft (2015) \cite{Soundarajan2015} show that detecting small groups of strongly connected nodes, according to a local association criterion, that is, in an ego approach, and the subsequent expansion of these small groups, produce communities more accurate than methods that only evaluate the structure of the entire network. This finding justifies the number of works that are dedicated to detecting communities in ego networks and to understand their structure and evolution.

McAuley and Leskovec (2012) \cite{McAuley:2012,Leskovec2012} address the problem of detecting communities by detecting what the authors consider as social circles of OSN users. The authors collected a set of 1000 ego networks from Twitter's follow interaction, each with a maximum of 4964 alters, plus some other Facebook and Google+ ego networks, all with manually labeled ground truth. The detection model combines the ego network structure and characteristics of the alters' profiles to group them into communities from the similarity between them. Among the characteristics used in the process of community formation, the authors do not consider the different types of interaction between the ego and the alters. 

The proposed model by McAuley and Leskovec is able to identify overlapping and hierarchical communities and it was elaborated in order to identify different similarity patterns for each community, which fits very nicely to the problem of identification of ego network communities, and would be impracticable in larger networks with thousands of us. The results show that the combination of the network structure with characteristics of the user profiles used by the proposed model was able to accurately identify the communities in a diverse set of data from Facebook, Google+ and Twitter. The application suggestion is to create a tool that uses the proposed template for automatic creation of Twitter lists.

For Dykstra and Lijffijt (2014) \cite{Dykstra2014} the problem of detecting communities in ego networks can be realized by predicting communities from a collection of duly labeled and collected data about other user communities. They propose the use of a database of lists collected from various users on Twitter to predict which lists fit best according to the structure of a particular user's ego network. The authors collected information on Twitter follow and follower interactions to model a total of 24 ego networks used in the experiment.

In the work of Epasto et al. (2015) \cite{Epasto2015} authors use Twitter's follow interaction to model ego networks and detect communities in those networks. The purpose of the paper is to identify similar alters that co-occur in different communities for recommendation of friendship between them within the OSN. Again we note that other types of user interaction are ignored at the expense of the explicit friendship relationship.

In the described works that detect communities in ego networks in OSN, we observe that the ego networks modeling is done only based on static information, such as the explicit friendship relationship defined in each OSN and information extracted from the users profile. Note that there is a very large amount of information present in other types of interaction, mainly because they are considered dynamic and performed with different intensities, which are simply ignored. The modeling proposed in this dissertation can be used as a way to verify if other types of information derived from dynamic interactions can also produce accurate communities.

%Não se encaixa no contexto de trabalhos correlatos...
%Papers that attempt to detect communities (lists) on Twitter ego networks, whether for the recommendation of lists or friends, have often used just the follow interaction to model the networks. We conjecture that the use of another type of interaction will be able to produce similar or even better results, since lists can be created as a way for the ego to receive content from a specific group of users that are not necessarily part of the friendship circle of the ego, or that simply the ego did not want to be identified as a follower of those alters. In this sense, the structural analysis of ego networks modeled based on information from other types of user interactions can change the perspective of the existing works in the literature that use only one type of interaction as the basis for the detection of communities in OSN.

% - Não trabalha análise estrutural - apresenta uma ferramenta para identificar padrões em grafos.
%In the work of Moustafa et al. (2012) \cite{Moustafa:2012} the authors present a new interface in the ego network analysis process, which may facilitate the process of finding patterns present in graphs, proposing a declarative language based on SQL. The queries made in this language should allow the display of patterns present in the relationships between the elements of an ego network, such as the node degree or its local grouping coefficient, for example. The data retrieved by the query can still be used in later analysis. The goal is to compare the graph under analysis with known structures that are stored in a database by performing a query through the SQL language. The authors also emphasize that this model can be useful in many domains in the analysis of social networks, including identification of opinion leaders, classification of nodes, prediction of links and identification of functions.


\section{Structural Analysis of Multilayer Network}
\label{sec:multilayer_net_analysis}

The analysis of multilayer networks passes through the understanding of the structure of the networks observing different relationships existing between the vertices. In the literature there are works that present a study and, generally, the comparison between the structures of each layer \cite{Wilson2009,Kwak:2010,Szell2010,Cardilo2013,Nicosia2015,Amati2015,Darmon2015,Omodei2015,DeArruda2016,Hong2016}, and works that analyze a single network structure formed by the aggregation of all the layers \cite{Boccaletti2014,Azaza2015,DeDomenico2015,Kuncheva2015,Hajibagheri2016,Azaza2016,Kleineberg2016,Vahedian:2017,Ding2017}. Here we present a brief description of some works that make a comparative analysis between structures found in different layers of the same multilayer network. The results of these works indicate the need to evaluate the formation process of the layers and the nature of the links between the elements of the network, according to what we are proposing in this dissertation.

In Wilson et al. (2009) \cite{Wilson2009}, the authors collected data from Facebook and modeled two networks, one with static reciprocal network friendship interaction and another with dynamics interactions between users. The comparison between these two networks showed that the dynamics interaction network presents larger diameters, lower clustering coefficients, and higher assortativity. The authors also report that only a small number of users maintain interaction activities, and that the number of interactions is limited by constraints, such as observation time, which calls into question the evaluation of OSNs only with the modeling of interactions dynamics. Even so, they strongly suggest that OSNs based applications are also designed with graphs of dynamic interactions and not only with static interactions.

Kwak et al. \cite{Kwak:2010} collect the whole network of Twitter and did an analysis of the topological characteristics through the type of interaction "follow" and the process of information diffusion. The main results of the analysis show that degree distribution did not follow a power law, diameter and reciprocity were low, countering what is expected in a social network in the real world. In another analysis the authors showed that the rankings of number of followers and number of retweets were different. Another important feature was the fact that a tweet that is retried reaches an average of 1,000 users, regardless of the number of followers of the original tweet author. This fact shows the power process of disseminating information on Twitter and the importance of retweet interaction.

According to Szell et al. (2010) \cite{Szell2010} the multidimensional nature of many social networks has been ignored, with the researchers giving emphasis to the study of only the topology of these networks, mainly due to the lack of available data. In their work the authors analyzed a complete multilayer social network, formed by the players of an online multiplayer game. By analyzing the network layers independently they found that they differ in relation to reciprocity, clustering coefficient and degree distribution, indicating that the combination of layers in a multilayer network may cause a poor representation of the system, hiding patterns presented in each layer.

%\cite{Szell2010}
%In their work the authors analyzed a complete multilayer social network, formed by the players of an online multiplayer game. Six types of interactions were modeled as layers of this multilayer network and classified into positive interactions (friendship, communication and trade) and negative interactions (enmity, aggression and punishment). The comparison between the layers showed that the negative interactions differ from the positive interactions by having lower reciprocity, weak clustering and long tail degree distribution. The authors further evaluated how the interdependence of different types of layers interferes determines the organization of the system. By studying the correlation and the overlap between different types of interactions, the results suggest that users play different roles in different layers.

In the work of Cardillo et al. (2012) \cite{Cardilo2013} the authors analyzed the structural properties of a multilayer network from flight data between airports in Europe. Flights operated by each airline represent one layer in the multilayer network. The results obtained by them indicate that topological properties of the whole system are not found in isolated layers, emphasizing the importance of evaluating the structural characteristics of each layer to understand the dynamic processes that occur in the aggregate network. The measures used in the topological analysis were the distribution of degree, average path lenght, coefficient of clustering, the size of the giant component and the Rich-club coefficient\footnote{Measures the tendency of highly connected nodes, i.e. the hubs, to be connected among themselves.}. Similar results were found by Hong et al. (2016) \cite{Hong2016} in a multilayer representation of the Chinese air transportation network (ATMN).

%In the work of Cardillo et al. (2012) \cite{Cardilo2013} authors question whether previous work evaluating structural features of complex networks is characterizing the unique layers. To answer this question they have modeled and analyzed the structural properties of a multilayer network from flight data between airports in Europe. Flights operated by each airline represent one layer in the multilayer network. The results obtained by them indicate that topological properties of the whole system are not found in isolated layers, but are representations of an emergent phenomenon closely related to the multilayer character of the system. The measures used in the topological analysis were the distribution of degree, average path lenght, coefficient of clustering, the size of the giant component and the Rich-club coefficient\footnote{Measures the tendency of highly connected nodes, i.e. the hubs, to be connected among themselves.}. In experiments where some layers are merged, according to structural similarity, as low cost airlines compared to large companies, it leads to the emergence of qualitatively different aggregate networks. And the authors emphasize the importance of evaluating the structural characteristics of each layer to understand the dynamic processes that occur in the aggregate network.

In \cite{Omodei2015} interactions between Twitter users were analyzed during exceptional events that related to politics, culture or science. For each event, \emph{retweets}, \emph{mentions} and \emph{reply} data were collected and a multilayer network constructed according to these activities. An analysis of each layer showed that between them there are significant differences between some structural properties, such as degree distribution and clustering coefficient. These results obtained in specific networks arouse the interest in carrying out a deeper analysis, such as the one we are proposing.

The detection of communities in OSN also began to be studied in multilayers networks. In \cite{Darmon2015} the authors performed the process of detecting communities on Twitter according to the type of action performed by the users. The generated communities were classified in: based on the relationship, the activity and the interactions between the users. The results showed that there are situations where the structures of these communities are similar, as there are also situations where the structures are completely different. Here the authors still mixed some interactions to model the three classes of networks studied. We intend to perform an analysis of the structure of each type of interaction in isolation.

Nicosia and Latora (2015) \cite{Nicosia2015} conducted a comparative study between layers of the same multilayer network to verify correlation between node activities and their degrees. The data set used is composed of real-world biological, technological, and social systems and spanning a wide range of sizes. The results show that real-world networks exhibit non-trivial multilayer correlations, which indicates that a representation of an aggregate network formed from the sum of its layers hides specific patterns of each layer. The authors caution that the role played by each layer's patterns may be responsible for completely new, unobservable physical phenomena in single-layer projections. Similar results were found by De Arruda et al. (2015) \cite{DeArruda2016} in a multilayer representation of an airport transport network.

A comparative structural analysis between layers modeled according to different types of interaction in a multilayer Twitter network was performed by Amati et al. (2015) \cite{Amati2015}. The dataset used is formed by data collected from users who are part of the public administration activities of Italy and its followers, with profiles in Italian language. From the metrics like degree distributions, connected components, path length distributions and clustering coefficients, the authors show the layers that correspond to the interaction of the mention type and retweets has higher information spreadind, compared to the layer of the follow interaction. Our work differs from Amati's research by performing a comparative structural analysis on a large number of ego networks.

Most of the studies on how to identify patterns in online social networks do not present conclusive results on the best way to treat the multilayer nature of these networks. We also did not find work evaluating the properties of multilayers networks using an egocentric approach to Twitter users. In this work we present results for metrics that evaluate the structural properties of each layer in a set of Twitter ego-networks, which is convenient for detecting repetitions of patterns or tendencies in the whole network.


%%%%%%%%%%%%%%%%%%%%%%%%%%%%%%%%%%%%%%%%%%%%%%%%%
% Conteúdo do Artigo... analisar para ver se encaixa em redes agregadas.
%%%%
%The prediction of Twitter links presented in \cite{Hajibagheri2016} shows that a more accurate result can be found by considering the properties of a multiplex network, rather than analyzing each layer in isolation.

%The combination of \emph{retweet}, \emph{reply}, \emph{reintroduce}, and \emph{read} actions among Twitter users forms a multilayer network that is explored in \cite{Zhaoyun2013}. The authors perform a sequence of random walks in the combined network to determine the influence of users, achieving convergence after four rounds.

%Detecting influential users on Twitter is also the purpose of the work \cite{Azaza2015} and \cite{Azaza2016}. The authors collect data on the European Parliament elections in 2014 and build a multilayer network combining the weight of user interactions in each layer: \emph{retweets}, \emph{mentions} and \emph{reply}. A general measure that quantifies the combination of different interactions is used to determine the degree of influence of a user.
