\chaves{Twitter, An\'alise Estrutural, Redes Ego Multicamadas, Detec\c c\~ao de Comunidades.}

\begin{resumo}

O Twitter \'e uma rede social online (RSO) que permite diferentes tipos de  intera\c{c}\~{a}o entre os usu\'arios (ex. ``seguir'', ``retuitar'', ``gostar'', ``mencionar'') e cada tipo de intera\c{c}\~{a}o possui um significado diferente. Entretanto, a maioria do trabalhos que modelam o Twitter como um grafo consideram apenas um tipo de liga\c{c}\~{a}o entre os usu\'{a}rios, notadamente, a intera\c{c}\~{a}o ``seguir''. Al\'{e}m disso, a an\'{a}lise macrosc\'{o}pica  de um grafo  grande como uma RSO  n\~{a}o revela caracter\'{i}sticas particulares do relacionamento dos v\'{e}rtices com suas vizinhan\c{c}as. Nesse  trabalho utilizamos de redes ou grafos ego multicamada para melhor modelar o as intera\c{c}\~{o}es entre um usu\'{a}rio  (o ego) e outros usu\'{a}rios do Twitter.
%O estudo  de v\'{a}rias  redes ego permite detectar repeti\c c\~oes de padrões de comportamento de interação a nível pessoal. A detecção desses  padrões são úteis   para melhorar  servi\c cos personalizados como recomenda\c~{a}o de tweets ou de usu\'{a}rios interlocutores e também para melhorar importantes aplicações em redes sociais como detec\c{c}\~{a}o de comunidades, an\'{a}lise da difus\~{a}o da informa\c{c}\~{a}o, entre outros.
Cada camada (subgrafo da rede ego) corresponde a um tipo espec\'{i}fico de intera\c{c}\~{a}o.  Consideramos  quatro tipos de camadas: {\em seguir}, {\em retuitar}, {\em gostar} e {\em mencionar}. A an\'{a}lise estrutual em 500 redes ego multicamadas nos permitiu verificar interessantes características das redes ego multicamadas do Twitter, tais como: (i) h\'{a} consider\'{a}vel interse\c{c}\~{a}o entre camadas (em termos de v\'{e}rtices e arestas), embora as camadas sejam diferentes entre si; (ii)  a grande maioria das camadas s\~{a}o ``small world'', i.e., a m\'{e}dia dos caminhos m\'{i}nimos \'{e} pequena, mas os coeficientes de agrupamento s\~{a}o altos; (iii) a distribui\c{c}\~{a}o de graus de entrada dos v\'{e}rtices  segue uma lei de pot\^{e}ncia cujo expoente varia entre camadas; (iv) a reciprocidade \'{e} alta na camada ``mencionar'' (superior a 0,4), mas baixa nas demais;
(v) todas as camadas t\^{e}m igual potencial para formarem comunidades.  Os resultados indicam que todas as camadas e as interese\c{c}\~{o}es entre elas possuem caracter\'{i}sticas importantes,   as quais nos  permite sugerir futuras investiga\c{c}\~{o}es nas \'{a}reas de detec\c{c}\~{a}o de comunidades, difus\~{a}o da informa\c{c}\~{a}o  e recomenda\c{c}\~{a}o de tu\'{i}tes. N\~{a}o se tem conhecimento de outro trabalho  que tenha investigado redes ego multicamadas no Twitter. 
\end{resumo}

