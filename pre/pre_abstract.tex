\keys{Twitter, Structural Analysis, Multilayer Ego  Networks,  Community Detection.}

\begin{abstract}{A Structural Analysis of Twitter Multilayer Ego Networks}


Twitter is an  online social  network (OSN)  that allows users to interact with each other by different interaction types (e.g., ``follow'', ``retweet'', ``like'', ``mention'') and each interaction type has a distinct meaning. However, most researchers that analyze Twitter as graph take into consideration only one interaction type  among users, mainly the ``follow'' interaction. Besides, a macroscopic analysis of a huge graph like an OSN does not reveal intrinsic characteristics of the relationships in a vertex  neighborhood. In this work, we use multilayer ego networks or multilayer ego graphs to better model the interactions between a user (the ego) and other users in Twitter.
Each layer  (a subgraph of the ego network) corresponds to a specific interaction type. We consider  four layer types: {\em follow}, {\em retweet}, {\em like} and  {\em mention}. A structural analysis of 500 multilayer ego networks allowed us to verify interesting characteristics of multilayer ego networks in Twitter, such as: (i) there is considerable overlap between layers (in terms of vertices  and edges), although layers differ from each other; (ii) most layers are    ``small world'', i. e., the mean shortest path in each layer is small, whereas the mean clustering coefficient is large; (iii) indegree distributions follow a power law, with exponent varying among layers; (iv) reciprocity between vertices is high in the follow layer (superior to 0.4), while smaller in the other layers; (v) all the four layers have similar potential to form communities. 
The results of our analysis indicate that all layers and the overlap between them have important characteristics which allowed for suggesting future investigations in the subjects  community detection, information diffusion and retweet recommendation. We do not know of any previous work which has investigated the structural properties of multilayer ego network in Twitter.
\end{abstract}
